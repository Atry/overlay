%%
%% Onward! submission using ACM SIGPLAN acmart format
%%
\documentclass[manuscript,anonymous,review,10pt]{acmart}

\AtBeginDocument{%
\providecommand\BibTeX{{Bib\TeX}}}

%% Disable ACM-specific metadata for submission
\settopmatter{printacmref=false}
\renewcommand\footnotetextcopyrightpermission[1]{}
\pagestyle{plain}

\usepackage{mathtools}

%% Conditional citation: hide self-citations under anonymous review
\newcommand{\selfcite}[2]{%
  \if@ACM@anonymous{#2}\else{#1}\fi
}

\newcommand{\Path}{\mathtt{Path}}
\newcommand{\dom}{\mathtt{dom}}
\newcommand{\properties}{\mathtt{properties}}
\newcommand{\overlays}{\mathtt{overlays}}
\newcommand{\supers}{\mathtt{supers}}
\newcommand{\references}{\mathtt{references}}
\newcommand{\ownproperties}{\mathtt{ownproperties}}
\newcommand{\bases}{\mathtt{bases}}
\newcommand{\resolve}{\mathtt{resolve}}
\newcommand{\this}{\mathtt{this}}
\newcommand{\snoc}{\mathbin{\triangleright}}
\newcommand{\init}{\mathtt{init}}
\newcommand{\last}{\mathtt{last}}
\newcommand{\self}{\mathbf{this}}

\begin{document}

\title{Overlay-Calculus}

\author{Bo Yang}
\affiliation{%
  \institution{Figure AI Inc.}
  \city{San Jose}
  \state{California}
  \country{USA}
}
\email{yang-bo@yang-bo.com}
\thanks{This work was conducted independently prior to the author's employment at Figure AI.}

\begin{abstract}
  Just as the $\lambda$-calculus uses three primitives (abstraction,
  application, variable) as the foundation of functional programming,
  Overlay-Calculus uses three primitives (record, reference, composition)
  as the foundation of declarative programming. Unlike the $\lambda$-calculus,
  which requires first-class functions, or Turing machines, which require
  mutable state, Overlay-Calculus is purely declarative and function-free,
  yet Turing complete. Composition ($\sqcup$) is commutative, idempotent,
  and associative---the linearization problem familiar from traditional
  mixin and trait systems simply does not arise. We demonstrate Turing
  completeness by giving a straightforward translation from the
  $\lambda$-calculus into Overlay-Calculus, showing that function application
  is derivable from symmetric overlay composition.
  Programs are inherently in A-normal form; unlike compiler-internal ANF
  which targets a fixed monad, Overlay-Calculus ANF supports dependency
  injection, making programs CPS-agnostic---the same code serves as either
  continuation-passing or direct-style depending on assembly.
  Records serve as tries, making Overlay-Calculus an immutable analogue of
  the RAM machine. These observations suggest Overlay-Calculus as a computational model
  for declarative programming, with applications to configuration languages,
  mixin-based object systems, composable effect systems, and modular software
  architectures.
\end{abstract}

\begin{CCSXML}
  <ccs2012>
  <concept>
  <concept_id>10003752.10010124.10010131.10010133</concept_id>
  <concept_desc>Theory of computation~Denotational semantics</concept_desc>
  <concept_significance>100</concept_significance>
  </concept>
  <concept>
  <concept_id>10003752.10010124.10010125.10010128</concept_id>
  <concept_desc>Theory of computation~Object oriented constructs</concept_desc>
  <concept_significance>500</concept_significance>
  </concept>
  <concept>
  <concept_id>10003752.10010124.10010125.10010127</concept_id>
  <concept_desc>Theory of computation~Functional constructs</concept_desc>
  <concept_significance>300</concept_significance>
  </concept>
  <concept>
  <concept_id>10003752.10010124.10010125.10010129</concept_id>
  <concept_desc>Theory of computation~Program schemes</concept_desc>
  <concept_significance>100</concept_significance>
  </concept>
  <concept>
  <concept_id>10011007.10011006.10011039</concept_id>
  <concept_desc>Software and its engineering~Formal language definitions</concept_desc>
  <concept_significance>500</concept_significance>
  </concept>
  <concept>
  <concept_id>10011007.10011006.10011008.10011009.10011019</concept_id>
  <concept_desc>Software and its engineering~Extensible languages</concept_desc>
  <concept_significance>500</concept_significance>
  </concept>
  <concept>
  <concept_id>10011007.10011006.10011008.10011009.10011011</concept_id>
  <concept_desc>Software and its engineering~Object oriented languages</concept_desc>
  <concept_significance>100</concept_significance>
  </concept>
  </ccs2012>
\end{CCSXML}

\ccsdesc[100]{Theory of computation~Denotational semantics}
\ccsdesc[500]{Theory of computation~Object oriented constructs}
\ccsdesc[300]{Theory of computation~Functional constructs}
\ccsdesc[100]{Theory of computation~Program schemes}
\ccsdesc[500]{Software and its engineering~Formal language definitions}
\ccsdesc[500]{Software and its engineering~Extensible languages}
\ccsdesc[100]{Software and its engineering~Object oriented languages}

\maketitle

\section{Introduction}

Declarative and configuration languages are ubiquitous in modern software
engineering. Systems such as NixOS modules~\cite{dolstra2008nixos,nixosmodules},
Jsonnet~\cite{jsonnet}, Hydra~\cite{hydra2023}, CUE~\cite{cue2019}, Dhall~\cite{dhall2017},
Kustomize, and JSON Patch (RFC~6902) all provide mechanisms for composing
structured data through inheritance or overlay. Among these, the NixOS
module system stands out: its recursive attribute set merging with
fixed-point semantics and deferred modules~\cite{nixosmodules} achieves
remarkable expressiveness in practice, and the mechanism has been adopted
well beyond NixOS itself---Home Manager~\cite{homemanager} manages user
environments across platforms, nix-darwin~\cite{nixdarwin} brings
declarative configuration to macOS, disko~\cite{disko} handles disk
partitioning, flake-parts~\cite{flakeparts} structures Nix flakes,
dream2nix~\cite{dream2nix} automates multi-language packaging,
devenv~\cite{devenv} composes developer environments, and
KubeNix~\cite{kubenix} and nixidy~\cite{nixidy} manage Kubernetes
clusters---all built on the same module system. Yet no computational
theory explains \emph{why} this mechanism is so powerful.

The $\lambda$-calculus serves as the foundational computational model for
functional programming. No analogous calculus exists for declarative
programming. This gap matters because the two paradigms differ in
fundamental ways. Configuration languages are inherently declarative:
their values are \emph{immutable} and they have \emph{no first-class
functions}. At first glance, Turing completeness appears incompatible
with these constraints. The three classical models of computation each
violate at least one of them:

\begin{center}
  \begin{tabular}{lccc}
    \textbf{Model} & \textbf{Turing complete} & \textbf{Immutable} & \textbf{No $\lambda$} \\
    \hline
    Turing Machine     & $\checkmark$ & $\times$     & $\checkmark$ \\
    $\lambda$-calculus & $\checkmark$ & $\checkmark$ & $\times$ \\
    RAM Machine        & $\checkmark$ & $\times$     & $\checkmark$ \\
  \end{tabular}
\end{center}

\noindent
The Turing Machine and RAM Machine require mutable state;
the $\lambda$-calculus requires first-class functions.
There was no known computational model that is simultaneously
Turing complete, immutable, and function-free. Yet the NixOS module
system already demonstrates that this gap can be bridged in practice:
its inheritance-based composition over recursive records, without
explicit functions, is expressive enough to configure entire operating
systems.

Conventionally, the value domain of configuration languages is assumed
to consist of finite, well-founded structures---initial algebras in the
sense of universal algebra. We challenge this assumption: configuration
values are better understood as lazily observable, possibly infinite
structures---formally, F-coalgebras~\cite{rutten2000universal}---where
semantics is determined by the observer and a finite prefix suffices
for any finite observation. This is not merely a theoretical
distinction: the \texttt{nixpkgs} package
collection~\cite{dolstra2006purely}---over 100{,}000
packages---is a lazy-evaluated F-coalgebra in practice, allowing any
single package to be evaluated without materializing the entire set.

This paper provides the missing theoretical foundation. We present
\textbf{Overlay-Calculus}, a minimal calculus that distills the essence of
this mechanism into three constructs---record literals, references,
and composition ($\sqcup$)---without functions or let-bindings.
A key observation is that
\emph{function application admits a straightforward translation into inheritance}:
a $\lambda$-abstraction corresponds to a record with an
$\mathrm{argument}$ slot and a $\mathrm{result}$ slot, and function
application corresponds to composing the function record with a record
that supplies the argument. This translation demonstrates that Overlay-Calculus
subsumes the $\lambda$-calculus and is therefore Turing complete.
This translation reveals an \emph{asymmetry}:
while embedding the $\lambda$-calculus into Overlay-Calculus is trivial,
the reverse is non-trivial because Overlay-Calculus provides primitive
operations---self-referential records, open recursion, symmetric
composition---that have no direct counterparts in the $\lambda$-calculus.
Moreover, Overlay-Calculus naturally encodes random-access memory:
records are tries, and composition is trie union, yielding an
immutable analogue of the RAM machine
(Section~\ref{sec:trie}).
This asymmetry suggests that Overlay-Calculus is not merely a
syntactic variant of the $\lambda$-calculus but captures
different computational patterns.

A further observation concerns the relationship between Overlay-Calculus and
continuation-passing style (CPS). Because every intermediate result in
Overlay-Calculus has a name, programs are inherently in A-normal
form~\cite{flanagan1993essence}. However, unlike ANF in functional
language compilers---which targets a fixed monad (typically the identity
monad~\cite{moggi1991notions}) and cannot express CPS
transformations---Overlay-Calculus ANF supports dependency injection through
empty record slots. This makes programs \emph{CPS-agnostic}: the same
code can be interpreted as continuation-passing (when slots are filled by
an external orchestrator) or as direct-style (when slots are composed
locally). From a category-theoretic perspective, this means abstract
interfaces and concrete implementations are syntactically
identical---only the assembly strategy differs. The $\lambda$-calculus
lacks this flexibility: converting between CPS and direct style requires
an explicit program transformation~\cite{plotkin1975callbyname,sabry1993reasoning}.

As a demonstration of this expressiveness, we port the Expression
Problem~\cite{wadler1998expression} solution of Wang and
Oliveira~\cite{wang2016trivially} from Scala (whose type system rests
on the DOT calculus~\cite{amin2016dot}) to Overlay-Calculus. The resulting
solution is even more concise than the original, despite Overlay-Calculus
having only three primitives whereas DOT is a substantially larger
system.

Another observation concerns the linearization problem that has long plagued
mixin-based systems~\cite{bracha1990mixin,c3linearization}. In traditional
mixin or trait calculi~\cite{scharli2003traits,ducasse2006traits}, composing
mixins that define the same property raises conflicts that must be resolved by
a linearization order. In Overlay-Calculus, because there are no scalar values,
the observable behavior at every node is solely \emph{property presence}---a set
of labels. Composition ($\sqcup$) is therefore commutative, idempotent, and
associative. The linearization problem does not arise---not because it is
resolved, but because it does not exist.

Just as the $\lambda$-calculus uses three primitives (abstraction,
application, variable) to serve as the foundation of functional
programming, Overlay-Calculus uses three primitives (record, reference,
composition) to serve as the foundation of declarative programming.

All constructions in this paper have been implemented and tested in
the Overlay language, an executable implementation of Overlay-Calculus using YAML syntax,
\selfcite{available as open source~\cite{mixin2025}}{included as supplementary material}. The implementation comprises a core
evaluator, a standard library of Church-encoded data types, and a test
suite covering every example in this paper: the Expression Problem
solution (Section~\ref{sec:expression-problem}), escape and multi-exit
continuations (Section~5.3), the Selector and Rebuilder combinators for
dynamic trie operations (Appendix~\ref{app:trie}), and Church-encoded
booleans and natural numbers (Appendix~\ref{app:church}). All tests are
mechanically verified against expected output snapshots.

\paragraph{Contributions.}
\begin{itemize}
  \item We present Overlay-Calculus, a minimal computational model for
    declarative programming that contains only overlay constructs (record,
    reference, composition) and no functions or scalar types.
  \item We observe that composition ($\sqcup$) is commutative, idempotent,
    and associative, and provide an intuitive explanation for why this
    holds. The linearization problem inherent in prior mixin and trait
    calculi simply does not arise.
  \item We give a translation from the $\lambda$-calculus to Overlay-Calculus,
    demonstrating that Overlay-Calculus subsumes the $\lambda$-calculus
    and is Turing complete.
    As a corollary, standard data types (booleans, natural numbers)
    can be Church-encoded without scalars (Appendix~\ref{app:church}).
  \item We show that Overlay-Calculus programs are CPS-agnostic: the same code
    can be interpreted as either continuation-passing or direct-style depending
    on assembly strategy, in contrast to the $\lambda$-calculus where CPS
    conversion is an explicit program transformation.
\end{itemize}

\paragraph{Informal example.}\label{sec:expression-problem}
The Expression Problem~\cite{wadler1998expression}
asks how to extend a data type with both new cases and new operations,
without modifying existing code.
In a typed setting this additionally requires static type safety;
Overlay-Calculus is untyped, so we focus on the extensibility aspect.
Suppose a standard library provides a $\mathrm{binnat}$ module for
binary natural arithmetic (see Appendix).
We define an expression language with an evaluation operation:
\begin{align*}
  \{ \quad \mathrm{expression} &\mapsto \{
      \mathrm{Constant} \mapsto \{\mathrm{magnitude} \mapsto \{\}\},\;
      \mathrm{Addition} \mapsto \{\mathrm{left} \mapsto \{\},\;
    \mathrm{right} \mapsto \{\}\}\}, \\[4pt]
    \mathrm{evaluation} &\mapsto \{
      \mathrm{Constant} \mapsto \mathrm{expression}.\self.\mathrm{Constant}
      \sqcup \{\mathrm{outcome} \mapsto
      \mathrm{Constant}.\self.\mathrm{magnitude}\}, \\
      &\qquad \mathrm{Addition} \mapsto
      \mathrm{expression}.\self.\mathrm{Addition} \sqcup \{ \\
        &\qquad\qquad
        \mathrm{left\_outcome} \mapsto \mathrm{Addition}.\self.\mathrm{left}.\mathrm{outcome},\\
        &\qquad\qquad
        \mathrm{right\_outcome} \mapsto \mathrm{Addition}.\self.\mathrm{right}.\mathrm{outcome},\\
        &\qquad\qquad
        \mathrm{sum} \mapsto \mathrm{binnat}.\self.\mathrm{Addition}
        \sqcup \{\mathrm{augend} \mapsto
        \mathrm{sum}.\self.\mathrm{left\_outcome},\;
        \mathrm{addend} \mapsto
        \mathrm{sum}.\self.\mathrm{right\_outcome}\}, \\
        &\qquad\qquad
    \mathrm{outcome} \mapsto \mathrm{Addition}.\self.\mathrm{sum}.\mathrm{outcome}\}\},\\[4pt]
    \mathrm{binnat} &\mapsto \{
      \mathrm{Addition} \mapsto
      \{\mathrm{augend} \mapsto \{\},\;
        \mathrm{addend} \mapsto \{\},\;
    \mathrm{outcome} \mapsto \{\}\}\}
  \quad \}
\end{align*}
A new operation (e.g., display) is added by defining a new
module that inherits from the same $\mathrm{expression}$ schemas:
\begin{align*}
  \mathrm{display} \mapsto \{
    \mathrm{Constant} &\mapsto \mathrm{expression}.\self.\mathrm{Constant}
    \sqcup \{\mathrm{representation} \mapsto
    \mathrm{Constant}.\self.\mathrm{magnitude}\},\\
    \mathrm{Addition} &\mapsto \mathrm{expression}.\self.\mathrm{Addition}
    \sqcup \{\mathrm{representation} \mapsto \{
        \mathrm{left} \mapsto
        \mathrm{Addition}.\self.\mathrm{left}.\mathrm{representation},\;
  \mathrm{right} \mapsto
  \mathrm{Addition}.\self.\mathrm{right}.\mathrm{representation}\}\}\}
\end{align*}
A new case (e.g., negation) is added with its operation handlers
and an empty API for the arithmetic primitive:
\begin{align*}
  \mathrm{negation} &\mapsto \{
    \mathrm{Negation} \mapsto
  \{\mathrm{operand} \mapsto \{\}\}\}, \\
  \mathrm{negation\_evaluation} &\mapsto \{
    \mathrm{Negation} \mapsto
    \mathrm{negation}.\self.\mathrm{Negation} \sqcup \{\\
      &\qquad
      \mathrm{operand\_outcome} \mapsto
      \mathrm{Negation}.\self.\mathrm{operand}.\mathrm{outcome},\\
      &\qquad
      \mathrm{negated} \mapsto \mathrm{binnat}.\self.\mathrm{Negation}
      \sqcup \{\mathrm{operand} \mapsto
      \mathrm{negated}.\self.\mathrm{operand\_outcome}\},\\
      &\qquad
  \mathrm{outcome} \mapsto
  \mathrm{Negation}.\self.\mathrm{negated}.\mathrm{outcome}\}\}, \\
  \mathrm{binnat} &\mapsto \{
    \mathrm{Negation} \mapsto
    \{\mathrm{operand} \mapsto \{\},\;
  \mathrm{outcome} \mapsto \{\}\}\}
\end{align*}
All modules compose freely via $\sqcup$; none requires modification of
existing definitions.
A complete executable version of this example---including both the
evaluation and display operations, a negation extension, and their free
composition---is \selfcite{available in the implementation~\cite{mixin2025}}{included in the supplementary material}, along with test
cases verifying evaluation and display for nested expressions mixing old
and new cases.

Note that $\mathrm{magnitude} \mapsto \{\}$ in $\mathrm{Constant}$
resembles a type annotation but is purely structural in Overlay-Calculus:
the evaluator works as long as the required properties appear somewhere in
the composition chain, regardless of any schema declarations.

\section{Syntax}

Let $e$ denote an expression.
Let $\ell$ denote a label (property name).
Let $k$ denote a non-negative integer (path length).

\begin{align*}
  e \quad ::= \quad & \{\ell_1 \mapsto e_1,\; \ldots,\; \ell_n \mapsto e_n\}
  && \text{(record, } n \ge 0\text{)} \\
  \mid\quad & \ell_{\mathrm{up}}.\self.\ell_{\mathrm{down},1}.\ell_{\mathrm{down},2}\ldots\ell_{\mathrm{down},k}
  && \text{(reference, } k \ge 0\text{)} \\
  \mid\quad & e_1 \sqcup e_2
  && \text{(composition)}
\end{align*}

The $\mapsto$ in $\{\ell \mapsto e\}$ defines a property, not a let-binding.
There is no variable binding in Overlay-Calculus.
In pursuit of a minimal computational model, Overlay-Calculus contains no
scalar types. This does not limit computational power: just as the
$\lambda$-calculus can encode booleans and natural numbers via Church
encoding, so can Overlay-Calculus (Appendix~\ref{app:church}). Practical
implementations built on Overlay-Calculus may include scalars and a
foreign-function interface; this does not affect the analysis in this
paper, as scalars can be treated as opaque sets whose composition
remains commutative and idempotent.

\paragraph{Qualified this.}
The reference $\ell_{\mathrm{up}}.\self.\ell_{\mathrm{down},1}\ldots\ell_{\mathrm{down},k}$
is analogous to Java's
\texttt{Outer.this.field}. Each $\{\ldots\}$ record has a label in its
enclosing scope; $\ell_{\mathrm{up}}$ names the target enclosing scope.
The \emph{qualified this} $\ell_{\mathrm{up}}.\self$ retrieves the \emph{dynamic self} of that scope---the
fully composed record after all $\sqcup$ merges have been applied.
The suffix $\ell_{\mathrm{down},1}\ldots\ell_{\mathrm{down},k}$ then projects properties from the
dynamic self.

A new scope level is created at each $\{\ldots\}$ (record literal).
The scope contains all properties of that overlay, including those introduced by $\sqcup$.
In particular, $\{a \mapsto \{\}\} \sqcup \{b \mapsto \{\}\}$ is a single overlay whose scope
contains both $a$ and $b$.
The operator $\sqcup$ does not create additional scope levels.

\begin{itemize}
  \item $\mathrm{Foo}.\self$ retrieves the dynamic self of the nearest
    enclosing record named $\mathrm{Foo}$.
  \item $\mathrm{Foo}.\self.\ell$ projects property $\ell$ from the
    dynamic self of $\mathrm{Foo}$.
  \item When $k = 0$, the reference $\ell_{\mathrm{up}}.\self$ inherits the entire
    enclosing scope, which may produce an infinitely deep tree.
\end{itemize}

\section{Overlay Trees}

Given an AST, what are the \emph{properties} visible at
each position in the tree? This section answers the question by
defining a collection of mutually recursive functions over paths.

Most of these functions have direct analogues in object-oriented languages:
$\Path$ identifies a position in the tree,
$\init$ and $\last$ navigate paths,
$\ownproperties$ extracts the locally defined members from the AST,
$\references$ extracts the inheritance declarations from the AST,
$\properties$ computes all visible members at a path including both locally defined and inherited ones,
$\supers$ computes the transitive inheritance closure,
and $\bases$ collects the direct base classes.
One function has no object-oriented analogue:
$\overlays$ collects the identity-sharing paths that arise
when composition ($\sqcup$) introduces multiple definitions
of the same label at the same scope level.
Two functions, the reference resolution function $\resolve$
and the qualified this resolution function $\this$,
have object-oriented analogues whose semantics differ significantly;
they are discussed in detail below.

All functions are pure and may be cached (memoised) without
changing semantics.

\paragraph{Path.}
A $\Path$ is a sequence of labels
$(\ell_1, \ell_2, \ldots, \ell_n)$ that identifies a position in
the tree.%
\footnote{Since paths are referentially transparent,
  an implementation may \emph{intern} paths so that
  structural equality reduces to pointer equality.}
We write $p$ for a path.
The \emph{root path} is the empty sequence $()$.
Given a path $p$ and a label $\ell$,
$p \snoc \ell$ is the sequence $p$ extended with $\ell$.
For a non-root path, the parent path $\init(p)$ is $p$ with
its last element removed,
and the final label $\last(p)$ is the last element of $p$.
For convenience, we sometimes write $\ell$ for $\last(p)$
when the path is clear from context.
With paths in hand, we can state what the AST provides
at each path.

\paragraph{AST.}
An expression $e$ (Section~2) is parsed into an AST.
The AST provides two primitive functions at each path $p$:
\begin{itemize}
  \item $\ownproperties(p)$: the set of labels
    that have subtrees at $p$.
  \item $\references(p)$: the set of reference pairs
    $(n,\; \ell_{\mathrm{down},*})$,
    where $n$ is the de~Bruijn index~\cite{debruijn1972lambda} (zero-based)
    and $\ell_{\mathrm{down},*}$ is the list of downward projections.
\end{itemize}

During parsing, a syntactic reference
$\ell_{\mathrm{up}}.\self.\ell_{\mathrm{down},1}\ldots\ell_{\mathrm{down},k}$
at path $p$ is resolved by finding the last occurrence of
$\ell_{\mathrm{up}}$ among the labels of $p$.
Let $p_{\mathrm{def}}$ be the prefix of $p$ up to and including
that occurrence.
The \emph{de~Bruijn index} $n$ is the number of labels in $p$
after that occurrence, i.e., $n = |p| - |p_{\mathrm{def}}|$.
The projections are
$\ell_{\mathrm{down},*} = (\ell_{\mathrm{down},1}, \ldots, \ell_{\mathrm{down},k})$.

Parsing populates $\ownproperties$ and $\references$ as follows:
\begin{itemize}
  \item A record $\{\ell_1 \mapsto e_1, \ldots, \ell_n \mapsto e_n\}$
    contributes $\{\ell_1, \ldots, \ell_n\}$ to $\ownproperties(p)$
    and nothing to $\references(p)$.
  \item A reference $\ell_{\mathrm{up}}.\self.\ell_{\mathrm{down},1}\ldots\ell_{\mathrm{down},k}$
    contributes one pair
    $(n,\; \ell_{\mathrm{down},*})$
    to $\references(p)$
    and nothing to $\ownproperties(p)$.
  \item A composition $e_1 \sqcup e_2$ merges both
    $\ownproperties$ and $\references$ from $e_1$ and $e_2$ at the same path.
\end{itemize}
Both $\ownproperties$ and $\references$ are pure data; they are not functions
of runtime state.
Given these two primitives, we can now answer the question
posed at the beginning of this section.

\paragraph{Properties.}
The properties of a path are its own properties together with
those inherited from all supers:
\[
  \properties(p) =
  \bigl\{\; \ell \;\big|\;
  (\_,\; p_{\mathrm{overlay}}) \in \supers(p),\;
  \ell \in \ownproperties(p_{\mathrm{overlay}})
  \;\bigr\}
\]
The remainder of this section defines $\supers$,
the transitive inheritance closure,
and its dependencies.

\paragraph{Supers.}
Intuitively, $\supers(p)$ collects every path that $p$ inherits from:
the identity-sharing paths $\overlays$ of $p$ itself,
plus the $\overlays$ of each
direct base $\bases$ of $p$, and so on transitively.
Each result is paired with the composition-site context
$\init(p_{\mathrm{base}})$ through which it is reached.
This provenance is needed by qualified this resolution, i.e.\ the $\this$ function defined below,
to map a definition-site overlay back to
the composition-site paths that incorporate it:
\[
  \supers(p) =
  \bigl\{\; (\init(p_{\mathrm{base}}),\; p_{\mathrm{overlay}}) \;\big|\;
  p_{\mathrm{base}} \in \bases^*(p),\;
  p_{\mathrm{overlay}} \in \overlays(p_{\mathrm{base}})
  \;\bigr\}
\]
The $\supers$ formula depends on two functions:
the identity-sharing paths $\overlays$ and the one-hop reference targets $\bases$.
We define $\overlays$ first.

\paragraph{Overlays.}
Intuitively, composition ($\sqcup$) can introduce multiple
definitions of the same label at the same scope level;
$\overlays(p)$ collects all such paths that share the
\emph{same identity} as $p$.
Concretely, the overlays of $p$ include $p$ itself
and $p_{\mathrm{branch}} \snoc \last(p)$ for every
branch $p_{\mathrm{branch}}$ of $\init(p)$ that also defines $\last(p)$:
\[
  \overlays(p) =
  \begin{cases}
    \{p\} & \text{if } p = ()  \\[6pt]
    \{p\}
    \;\cup\;
    \left\{\; p_{\mathrm{branch}} \snoc \last(p) \;\left|\;
    \begin{aligned}
      &(\_,\; p_{\mathrm{branch}}) \in \supers(\init(p)), \\
      &\text{s.t.}\; \last(p) \in \ownproperties(p_{\mathrm{branch}})
    \end{aligned}
    \right.\right\}
    & \text{if } p \neq ()
  \end{cases}
\]
It remains to define $\bases$, the other dependency of $\supers$.

\paragraph{Bases.}
Intuitively, $\bases(p)$ are the paths that $p$ directly
inherits from via references---analogous to the direct base
classes in object-oriented languages.
Concretely, $\bases$ resolves every reference in
$p$'s $\overlays$ one step:
\[
  \bases(p) =
  \left\{\; p_{\mathrm{target}} \;\left|\;
  \begin{aligned}
    &p_{\mathrm{overlay}} \in \overlays(p), \\
    &(n,\; \ell_{\mathrm{down},*}) \in \references(p_{\mathrm{overlay}}), \\
    &p_{\mathrm{target}} \in \resolve(\init(p),\; p_{\mathrm{overlay}},\; n,\; \ell_{\mathrm{down},*})
  \end{aligned}
  \right.\right\}
\]
The $\bases$ formula calls the reference resolution function $\resolve$, which we define next.

\paragraph{Reference resolution.}
Intuitively, $\resolve$ turns a syntactic reference into the
set of paths it points to in the fully composed tree.
It takes a composition-site path $p_{\mathrm{site}}$,
a definition-site path $p_{\mathrm{def}}$,
a de~Bruijn index $n$,
and downward projections $\ell_{\mathrm{down},*}$.
Resolution proceeds in two phases: first, the qualified this resolution function $\this$ performs $n$ upward steps
starting from the enclosing scope $\init(p_{\mathrm{def}})$,
mapping the definition-site path to composition-site paths;
then the downward projections $\ell_{\mathrm{down},*}$ are appended:
\[
  \resolve(p_{\mathrm{site}},\; p_{\mathrm{def}},\; n,\; \ell_{\mathrm{down},*})
  =
  \bigl\{\;
  p_{\mathrm{current}} \snoc \ell_{\mathrm{down},1} \snoc \cdots \snoc \ell_{\mathrm{down},k}
  \;\big|\;
  p_{\mathrm{current}} \in
  \this(\{p_{\mathrm{site}}\},\; \init(p_{\mathrm{def}}),\; n)
  \;\bigr\}
\]
When multiple routes exist (due to multi-path inheritance from $\sqcup$),
they may yield different target paths; all are collected.
We now define $\this$.

\paragraph{Qualified this resolution.}
Intuitively, $\this$ answers the question:
``in the fully composed tree, where does
the definition-site scope $p_{\mathrm{def}}$ actually live?''
The design of $\this$ reconciles two historically opposing
approaches to scope resolution.
Early Lisp implementations used dynamic scope~\cite{mccarthy1978history},
where variable references are resolved against the run-time call stack;
this is flexible but unpredictable.
Scheme~\cite{sussman1975scheme} corrected this with lexical scope,
where a de~Bruijn index resolves in a single step by indexing
into the statically determined environment;
this is predictable but rigid.
Scala~\cite{odersky2004overview} and the NixOS module
system~\cite{nixosmodules} follow the lexical approach:
each \texttt{this} reference is bound to a single
statically known path.
When composition introduces multiple inheritance routes
to the same scope, Scala rejects the ambiguity at compile time
and the NixOS module system enters an infinite loop---neither
can express that all routes are equally valid.

Overlay-Calculus combines the strengths of both approaches.
The reference itself is lexically determined at the definition site
as a de~Bruijn index, so what $\this$ looks up is predictable.
But because composition ($\sqcup$) interleaves scopes from
different overlays, the result depends on the composed tree
structure: $\this$ walks upward one scope level at a time,
consulting $\supers$ at each level, and naturally tracks a
\emph{set} of composition-site paths rather than a single one.
All routes contribute equally to the composed result, and since
$\sqcup$ is idempotent, duplicated contributions are harmless.
The only observable is property presence, making
multi-path resolution well-defined without disambiguation.

Concretely, each step finds, among the supers of every path in the
frontier $S$, those whose overlay component matches
$p_{\mathrm{def}}$, and collects the corresponding
composition-site paths as the new frontier.
After $n$ steps the frontier contains the answer:
\[
  \this(S,\; p_{\mathrm{def}},\; n) =
  \begin{cases}
    S & \text{if } n = 0 \\[6pt]
    \this\!\left(
    \left\{\; p_{\mathrm{site}} \;\left|\;
    \begin{aligned}
      &p_{\mathrm{current}} \in S, \\
      &(p_{\mathrm{site}},\; p_{\mathrm{overlay}})
      \in \supers(p_{\mathrm{current}}), \\
      &\text{s.t.}\; p_{\mathrm{overlay}} = p_{\mathrm{def}}
    \end{aligned}
    \right.\right\},\;
    \init(p_{\mathrm{def}}),\;
    n - 1
    \right)
    & \text{if } n > 0
  \end{cases}
\]
At each step $\init$ shortens $p_{\mathrm{def}}$ by one label
and $n$ decreases by one.
Since $n$ is a non-negative integer, the recursion terminates.

This completes the chain of definitions needed to compute $\properties(p)$.
We conclude this section with the algebraic properties of $\sqcup$.

\paragraph{Properties of $\sqcup$.}
Composition is commutative, idempotent, and associative.
The identity element is the empty record $\{\}$.

\medskip\noindent\textit{Intuition.}\quad
Evaluation merges $\references$ and $\ownproperties$ for each label recursively.
At every leaf the only observable is property presence---a set of labels.
Since there are no scalars, the only observable at any node is which properties exist.
Set union is commutative and idempotent:
$\{a,b\} \cup \{b,c\} = \{b,c\} \cup \{a,b\}$ and
$\mathrm{Labels} \cup \mathrm{Labels} = \mathrm{Labels}$.
The observable tree is therefore invariant under reordering or
duplication of references.

Since duplicate references produce the same observable result,
implementations may deduplicate references (e.g., by path)
to avoid the exponential blowup from diamond inheritance.

The only externally observable fact is property presence:
$\ell \in \properties(p)$.


\section{Relationship to Other Computational Models}

\subsection{Translation from $\lambda$-Calculus}

The following translation maps the $\lambda$-calculus to Overlay-Calculus.
Let $\mathcal{T}$ denote the translation function,
mapping a $\lambda$-term to an Overlay-Calculus expression.
Each $\lambda$-abstraction is assigned a unique label
$\mathrm{fn}_i$ in its enclosing scope.

\medskip
\begin{center}
  \begin{tabular}{l@{\qquad$\longrightarrow$\qquad}l}
    $x$ (variable) &
    $\mathrm{fn}_x.\self.\mathrm{argument}$ \\
    $\lambda x.\, e$ (abstraction) &
    $\{\mathrm{argument} \mapsto \{\},\;
    \mathrm{result} \mapsto \mathcal{T}(e)\}$ \\
    $e_1\; e_2$ (application) &
    $\mathcal{T}(e_1) \sqcup
    \{\mathrm{argument} \mapsto \mathcal{T}(e_2)\}$
  \end{tabular}
\end{center}
\medskip

A variable reference $x$ becomes
$\mathrm{fn}_x.\self.\mathrm{argument}$, where $\mathrm{fn}_x$
is the label of the $\lambda$-abstraction that binds $x$.
The qualified this $\mathrm{fn}_x.\self$ resolves to the dynamic
self of that abstraction's record, giving access to the
$\mathrm{argument}$ slot after all compositions.

The result of an application is accessed via $.\mathrm{result}$.
Nested applications require ANF-style naming of intermediate results.

The translation from $\lambda$-calculus to Overlay-Calculus is straightforward.
The reverse direction is non-trivial, as we discuss next.

\subsection{Expressive Asymmetry}

The translation above consists of three simple rules. This simplicity
is significant: the full computational power of the $\lambda$-calculus
embeds into Overlay-Calculus with minimal overhead.

The reverse direction---encoding Overlay-Calculus in pure
$\lambda$-calculus---is non-trivial.
Defining the meaning of Overlay-Calculus requires machinery for
lazy allocation, recursive data structures, and set-valued
resolution that have no direct counterparts in the $\lambda$-calculus.
This sophistication is not accidental---it reflects three features:

\begin{enumerate}
  \item \textbf{Self-referential records with lazy evaluation.}
    An overlay's properties can reference other properties and enclosing scopes
    through qualified this without explicit binding.
    In the $\lambda$-calculus, encoding such self-referential
    structures requires explicit allocation of mutable or lazy
    references (e.g., Haskell-style thunks or ML-style \texttt{ref}
    cells), obscuring the declarative intent.

  \item \textbf{Open recursion with symmetric composition.}
    The operator $\sqcup$ merges definitions from independent sources,
    all sharing the same composed overlay tree. Encoding this in
    the $\lambda$-calculus requires maintaining an extensible dictionary
    of methods that can be merged from multiple directions. Cook's
    denotational semantics of inheritance~\cite{cook1989denotational}
    devoted an entire dissertation to formalizing this pattern in a
    functional setting.

  \item \textbf{Commutative, idempotent merge.}
    Composition in Overlay-Calculus is symmetric:
    $e_1 \sqcup e_2$ and $e_2 \sqcup e_1$ produce the same observable
    result. In the $\lambda$-calculus, function composition
    $(f \circ g)$ is neither commutative nor idempotent, and simulating
    these properties requires additional machinery.
\end{enumerate}

\noindent
This asymmetry reveals that Overlay-Calculus is not simply the
$\lambda$-calculus in disguise. Rather, Overlay-Calculus provides
primitive operations for computational patterns that are
\emph{derivable but complex} in the $\lambda$-calculus.

\subsection{CPS-Agnostic Programs}

In the $\lambda$-calculus, a program is either in continuation-passing
style (CPS) or direct style. Converting between these representations
requires explicit transformation---the CPS
conversion~\cite{plotkin1975callbyname,sabry1993reasoning}.
Overlay-Calculus exhibits a different property: the same program text can
be interpreted as either CPS or direct-style depending on how
dependencies are assembled.

Consider a generic computation step:
\begin{align*}
  \mathrm{Step} \mapsto \{
    \mathrm{input} \mapsto \{\},\;
    \mathrm{process} \mapsto \{\},\;
    \mathrm{result} \mapsto \{\}
  \}
\end{align*}

The empty $\mathrm{process}$ slot is an \emph{interpretation point}. If
filled externally by an orchestrator, it acts as a continuation parameter
(CPS interpretation). If composed locally with an implementation, it acts
as a direct dependency (direct-style interpretation). The
$\mathrm{Step}$ code remains unchanged.

\paragraph{CPS interpretation via external orchestration.}
When $\mathrm{process}$ is injected from outside, the computation passes
its intermediate result to an external handler:
\begin{align*}
  &\mathrm{Step} \sqcup \{ \\
    &\quad \mathrm{input} \mapsto \{\mathrm{data} \mapsto \{\}\},\; \\
    &\quad \mathrm{process} \mapsto \{ \\
      &\qquad \mathrm{result} \mapsto \{\mathrm{handled} \mapsto \{\}\} \\
    &\quad \} \\
  &\}
\end{align*}
The $\mathrm{process}$ slot acts as a continuation: it receives the
result and determines what happens next. This is CPS-style programming.

\paragraph{Direct-style interpretation via local composition.}
Alternatively, $\mathrm{process}$ can be filled by local composition:
\begin{align*}
  &\mathrm{Step} \sqcup \{ \\
    &\quad \mathrm{input} \mapsto \{\mathrm{data} \mapsto \{\}\},\; \\
    &\quad \mathrm{process.result} \mapsto \{\mathrm{computed} \mapsto \{\}\} \\
  &\}
\end{align*}
Here $\mathrm{process}$ is implemented directly within the composition.
This is direct-style programming. The $\mathrm{Step}$ code is identical
in both cases---only the assembly strategy differs.

\paragraph{Representation independence.}
This property is a form of representation
independence~\cite{reynolds1983types} applied to control flow: client
code using $\mathrm{Step}$ need not know whether $\mathrm{process}$ will
be filled continuation-style or directly. Library authors write code
once; users choose the control flow style at assembly time.

The $\lambda$-calculus lacks this flexibility. A function written in
direct style must be explicitly transformed (CPS conversion) before it
can be used continuation-style. Overlay-Calculus programs are inherently in
A-normal form~\cite{flanagan1993essence}, where every intermediate result
has a name.

However, Overlay-Calculus ANF differs critically from ANF in functional
programming. In functional language compilers, ANF conversion requires
specifying a target monad---either a concrete monad or an abstract
one~\cite{moggi1991notions}. Compiler-internal ANF transformations (the
ANF form used in intermediate representations) typically target the
identity monad, which cannot express CPS transformations. This is
analogous to how imperative compilers use SSA (Static Single Assignment)
form, which similarly lacks control flow flexibility. Overlay-Calculus ANF,
by contrast, has built-in dependency injection: empty API slots can be
filled by external orchestrators. This capability enables the same ANF
program to serve both CPS and direct-style interpretations without monad
specification or transformation.

\paragraph{Multiple symmetric continuations.}
When interpreted continuation-style, Overlay-Calculus naturally supports
multiple exit points:
\begin{align*}
  \mathrm{Choice} \mapsto \{
    \mathrm{on\_success} &\mapsto \{\},\; \\
    \mathrm{on\_failure} &\mapsto \{\},\; \\
    \mathrm{on\_timeout} &\mapsto \{\},\; \\
    \mathrm{result} &\mapsto \{\}
  \}
\end{align*}

All continuation slots are symmetric under $\sqcup$. Traditional CPS in
the $\lambda$-calculus distinguishes a single continuation parameter;
additional continuations must be encoded (e.g., as pairs or disjoint
sums). Overlay-Calculus treats all continuation slots uniformly, eliminating
the need for specialized encodings.

\paragraph{Abstract and concrete are syntactically identical.}
From a category-theoretic perspective, abstract algebra and concrete
implementation have the same syntactic structure in Overlay-Calculus:
\begin{align*}
  \text{Abstract:} \quad & \mathrm{API} \mapsto \{\} \\
  \text{Concrete:} \quad & \mathrm{API} \mapsto \{\text{implementation}\}
\end{align*}

This is the abstract/concrete correspondence: the same functor applies to
both. In traditional languages, abstract interfaces (e.g., Java
interfaces, Haskell type classes) have different syntax from concrete
implementations. In Overlay-Calculus, they are syntactically
identical---only the interpretation differs. An empty API is
simultaneously an abstract requirement (when viewed as a dependency) and
a concrete extension point (when viewed as an injection site).

The Selector and Rebuilder combinators (Appendix~\ref{app:trie})
demonstrate this flexibility: their callback slots
($\mathrm{on\_zero}$, $\mathrm{on\_odd}$, $\mathrm{on\_even}$) can be
filled externally (CPS interpretation) or composed locally (direct
interpretation).

\paragraph{Practical implications.}
This design enables flexible library interfaces. A library component with
empty API slots can be used in multiple contexts without modification:
\begin{itemize}
  \item In synchronous code, fill slots with direct implementations
  \item In asynchronous code, fill slots with continuation handlers
  \item In effect systems, compose with algebraic effect handlers~\cite{plotkin2003algebraic}
  \item In test code, fill slots with mock implementations
\end{itemize}

No dual APIs are needed. The same component serves all use cases. This
eliminates a common source of code duplication in traditional languages,
where async and sync versions of the same functionality must be
maintained separately.

\selfcite{The implementation~\cite{mixin2025} includes}{The supplementary material includes} executable examples of all three
patterns described above: escape continuations analogous to
\texttt{call/cc}, success/failure continuations analogous to exception
handling, and protected computations with multiple symmetric exit points.
Each example includes test cases verifying that the correct continuation
path is taken.

\subsection{Records as Tries: The RAM Machine Connection}
\label{sec:trie}

A record $\{\ell_1 \mapsto e_1,\; \ldots,\; \ell_n \mapsto e_n\}$ is
a trie node: each label selects a subtree. Nested records form a
trie whose depth corresponds to key length. Composition ($\sqcup$) is
natively trie union: merging two records combines their subtrees
recursively. This observation yields an immutable analogue of
random-access memory.

The fundamental operations on random-access memory correspond directly
to Overlay-Calculus primitives:
\begin{itemize}
  \item \textbf{Write}: composing a singleton trie ($\sqcup$).
  \item \textbf{Read (static)}: qualified-this resolution
    ($\ell_{\mathrm{up}}.\self.\ell_{\mathrm{down},1}.\ell_{\mathrm{down},2}\ldots\ell_{\mathrm{down},k}$).
  \item \textbf{Read (dynamic)}: folding a key into a Selector
    combinator that navigates the trie at runtime.
  \item \textbf{Delete}: folding a key into a Rebuilder combinator
    that reconstructs the trie, omitting the targeted entry.
  \item \textbf{Update}: composing deletion with insertion.
\end{itemize}

\noindent
In the $\lambda$-calculus, implementing a persistent trie with
efficient union requires explicit data structure encodings
(e.g., hash array mapped tries or finger trees). In Overlay-Calculus,
records \emph{are} tries and composition \emph{is} trie union---no
additional data structure is needed.
The full constructions (Selector, Rebuilder, Lookup, Delete, Update)
are given in Appendix~\ref{app:trie}.
\selfcite{The implementation~\cite{mixin2025} includes}{The supplementary material includes} executable implementations of all
trie operations---static and dynamic lookup, deletion, update, and trie
union---with test cases covering keys of varying bit structure.

\subsection{Contrast with SKI Combinator Calculus}

A natural question arises: is Overlay-Calculus merely another function-free
reformulation of the $\lambda$-calculus, analogous to the SKI combinator
calculus~\cite{schonfinkel1924,curry1958combinatory}?

Like Overlay-Calculus, the SKI combinator calculus is function-free at the
object level (no $\lambda$-abstractions), Turing complete, and minimal
(three combinators). However, SKI is \emph{isomorphic} to the
$\lambda$-calculus in terms of problem difficulty: the bracket
abstraction algorithm translates between the two systems while
preserving computational structure. Problems that are hard in the
$\lambda$-calculus---such as the Expression
Problem~(Section~\ref{sec:expression-problem})---remain equally hard
in SKI.

Overlay-Calculus, by contrast, exhibits a genuine shift in problem
difficulty. The Expression Problem, which requires sophisticated
encodings in $\lambda$-calculus-based systems---object
algebras~\cite{oliveira2012extensibility}, finally tagless
interpreters~\cite{carette2009finally}, or multi-parameter type
classes---becomes direct and natural in Overlay-Calculus via $\sqcup$.
Similarly, random-access memory via immutable tries, which requires
explicit data structure engineering in the $\lambda$-calculus, is
simply the native record structure of Overlay-Calculus
(Section~\ref{sec:trie}). This difference in problem difficulty
demonstrates that Overlay-Calculus captures computational patterns that
are \emph{different} from those of the $\lambda$-calculus,
not merely syntactic variants.

\section{Discussion}

\subsection{Is Overlay-Calculus Just Functions in Disguise?}

A skeptical reader might ask whether Overlay-Calculus is simply
the $\lambda$-calculus with different syntax.

The semantic domain of Overlay-Calculus (Section~3) is entirely
first-order: paths are label sequences, overlays are the record trees
produced by evaluation, AST nodes are pure data,
and the resolution functions ($\overlays$, $\supers$, $\resolve$, etc.)
are ordinary set-valued functions over these structures.
No function types appear in the object language or its semantics.
The crucial question is not what notation is used, but what
computational patterns are \emph{primitive} in the language itself.

In Overlay-Calculus, the primitive operations are record construction,
qualified-this reference, and composition. Function application is
\emph{derived} from these primitives (Section~5.1), not primitive itself.
Indeed, the asymmetry demonstrated in Section~5.2 makes this concrete:
embedding the $\lambda$-calculus into Overlay-Calculus requires only three
simple rules, while the reverse direction requires sophisticated
encoding. The real test of whether two computational
models are ``the same'' is whether they make the same problems easy
and the same problems hard. As shown in Section~5, they do not:

\begin{center}
  \begin{tabular}{lll}
    \textbf{System} & \textbf{Expression Problem} & \textbf{Random-access memory} \\
    \hline
    $\lambda$-calculus
    & Object algebras~\cite{oliveira2012extensibility},
    & Explicit trie encoding \\
    & finally tagless~\cite{carette2009finally} & \\
    SKI combinators
    & Same difficulty as $\lambda$
    & Same difficulty as $\lambda$ \\
    Overlay-Calculus
    & Direct composition ($\sqcup$)
    & Records \emph{are} tries \\
  \end{tabular}
\end{center}

\noindent
The fact that Overlay-Calculus requires no additional machinery for either
problem class---while the $\lambda$-calculus requires elaborate
encodings for both---demonstrates that Overlay-Calculus captures
different computational primitives, not merely a
syntactic reshuffling.

\subsection{Properties of Composition}

Commutativity and idempotence of $\sqcup$ make Overlay-Calculus
transparent to linearization order.
Deduplication of inherited overlays is purely an optimization.

\subsection{Relationship to Practical Systems}

Overlay-Calculus provides a theoretical foundation for understanding several
classes of declarative systems that have emerged independently in practice.

\paragraph{Configuration languages.}
The NixOS module system~\cite{dolstra2008nixos,nixosmodules}---whose
recursive attribute merging motivated this work---can be understood as an
implementation of Overlay-Calculus with additional features (type checking,
error reporting, FFI to the Nix language). The module system's
expressiveness stems from the properties described in Section~3: composition
is commutative, idempotent, and associative, so the linearization
problem does not arise. Other configuration languages exhibit similar patterns:
Jsonnet~\cite{jsonnet} provides inheritance-based composition over JSON;
CUE~\cite{cue2019} unifies types and values through lattice-based merging;
Dhall~\cite{dhall2017} combines functional programming with configuration.
All share Overlay-Calculus's core pattern: composition over recursive
structures without explicit functions.

\paragraph{Mixin-based object systems.}
Scala traits~\cite{odersky2004overview}, Ruby mixins~\cite{flanagan2008ruby},
and similar object-oriented mechanisms use symmetric composition to extend
classes. Traditional mixin calculi~\cite{bracha1990mixin,scharli2003traits}
face the linearization problem when composing mixins with conflicting
properties. In Overlay-Calculus, this problem does not arise (Section~3) by eliminating
scalar values: only property presence is observable, making composition
naturally commutative. This explains why trait systems with
conflict-resolution strategies (e.g., Scala's linearization) require
additional machinery that Overlay-Calculus does not.

\paragraph{Effect systems.}
Algebraic effects with handlers~\cite{plotkin2003algebraic,pretnar2015introduction}
and monad transformers structure computational effects through composition.
The CPS-agnostic property (Section~5.3) shows how Overlay-Calculus naturally
supports effect handler composition: empty API slots can be filled with
effect handlers (CPS interpretation) or direct implementations (direct-style
interpretation). This correspondence suggests that Overlay-Calculus may provide
a foundation for understanding effect systems without requiring explicit
monad machinery.

\paragraph{Modular software architectures.}
Plugin systems, component frameworks, and dependency injection frameworks
enable software extensibility through declarative composition. The
symmetric, associative nature of $\sqcup$ (Section~3) explains why such
systems work: components can be composed in any order without changing
behavior. Overlay-Calculus formalizes the composition patterns that these
systems implement ad hoc.

\paragraph{Union file systems.}
Union file systems (UnionFS, OverlayFS, AUFS) layer multiple directory
hierarchies to present a unified view. The composition semantics directly
mirror Overlay-Calculus: later layers override earlier layers (asymmetric
composition), files from all layers remain accessible (non-destructive
merging), and composition order affects conflict resolution (non-commutative
when scalar files conflict). A proof-of-concept
\selfcite{implementation~\cite{ratarmount2025} (included as supplementary material)}{implementation (included as supplementary material)}
demonstrates that Overlay-Calculus
patterns emerge naturally in union mount composition through late-binding
semantics and dynamic dispatch for file lookups across layers.

\section{Future Work}

\selfcite{The Overlay language~\cite{mixin2025}}{The supplementary
implementation} is an executable implementation of Overlay-Calculus
that already goes beyond the untyped calculus presented here:
it includes compile-time checking that all references resolve to
valid paths in the composition, and a foreign-function interface (FFI)
that introduces scalar values from the host language.
The future work described below concerns the distance between this
implementation and a fully practical language---spanning both
theoretical foundations and library-level encodings.

\paragraph{Type system.}
The Overlay language's existing compile-time checks verify that every
reference path resolves to a property that exists in the composed
result. Formalizing the soundness of these checks---proving that
well-typed programs do not produce dangling references at
runtime---is one direction of future work.
Such a formalization would relate to the DOT
calculus~\cite{amin2016dot} in a manner analogous to how
Overlay-Calculus relates to the $\lambda$-calculus: a Typed
Overlay-Calculus can be viewed as DOT without $\lambda$, retaining
path-dependent types while replacing functions with
inheritance-based composition.
A formal type system would be amenable to mechanization in proof
assistants (Coq, Agda) for verifying type safety, progress, and
preservation properties.

Beyond soundness of the existing checks, additional type system
features would strengthen the language.
A key example is \emph{totality checking}: verifying that a
composition provides implementations for all required slots, not
merely that references resolve.
In untyped Overlay-Calculus, writing $\mathrm{magnitude} \mapsto \{\}$
is purely documentary---it defines a structural slot but imposes no
constraint on what must fill it.
Totality checking would enforce such declarations statically,
rejecting compositions that leave required slots unfilled.

\paragraph{Standard library.}
The remaining items concern the \emph{standard library}: they require
no extensions to the calculus or type system, but involve nontrivial
encodings within the existing framework.

Overlay-Calculus is \emph{open by default}: composition ($\sqcup$)
can freely merge any two overlays, and a value may simultaneously
inhabit multiple constructors (e.g.,
$\mathrm{Zero} \sqcup \mathrm{Odd}(\mathrm{Zero})$).
This is not a defect---it is the natural trie semantics of the
calculus, and the basis for solving the expression problem
(Section~5.2).
However, many operations (e.g., equality testing) assume that values
are \emph{linear}---inhabiting exactly one constructor.
The observer pattern (Appendix~\ref{app:church}) does not enforce
this: applied to a multi-constructor value, all callbacks fire and
their results are composed.

\emph{Schema validation} can be implemented at the library level
using existing primitives.
An observer can test whether a particular constructor is present,
returning a Boolean; a second Boolean dispatch then branches on the
result.
By chaining such tests, a factory can validate that a value matches
exactly one constructor and carries the required fields.
This does not restrict $\sqcup$ itself---composition remains
unconstrained---but provides a way to detect invariant violations
before they propagate.

The same technique---observer returning Boolean, followed by Boolean
dispatch---yields \emph{closed pattern matching}.
The observer pattern of Appendix~\ref{app:church} is inherently
\emph{open}: new constructor cases can be added through composition.
Closed matching, where exactly one branch is taken, can be encoded as
a \emph{visitor chain}: a linked list of if-then-else nodes, each
testing one constructor and falling through on mismatch.
This is analogous to GHC.Generics' $\mathrm{(:+:)}$ sum
representation~\cite{magalhaes2010generic}, where each link
corresponds to one summand.
An important consequence is that such chains do \emph{not}
commute---they have a fixed order---and therefore cannot be extended
through open composition, which is the correct semantics for closed
dispatch.

The trie operations in Appendix~\ref{app:trie} are hardcoded for
binary natural number keys.
Since every overlay \emph{is} a trie (Section~5.4) and insertion is
simply $\sqcup$, only operations that cannot be expressed as
composition need explicit implementation: \emph{deletion},
\emph{dynamic lookup}, and \emph{prefix lookup}.
Generalizing to arbitrary key types requires each key factory to
provide two interfaces---\emph{SelectTrieChild} (select a subtrie by
constructor) and \emph{ReplaceTrieChild} (replace a subtrie by
constructor while preserving the rest)---corresponding to the getter
and setter of a lens~\cite{foster2007combinators} focused on the
constructor-determined child.
Prefix lookup is structurally analogous to equality testing: both
synchronously traverse a key (a single-path trie) alongside another
structure, dispatching at each level.
The difference is the return type---a Boolean for equality, a subtrie
for prefix lookup.

\bibliographystyle{ACM-Reference-Format}
\bibliography{mixin-calculus}

\appendix

\section{Trie Operations}
\label{app:trie}

This appendix gives the full constructions for the RAM machine
operations summarized in Section~\ref{sec:trie}.
Index the trie by binary natural numbers using three labels---$\mathrm{value}$
for the stored entry at a node, $\mathrm{at\_odd}$ and $\mathrm{at\_even}$
for the two subtrees corresponding to the bit structure of the key:
\begin{align*}
  \mathrm{Zero} &\;\longrightarrow\; \mathrm{value} \\
  \mathrm{Odd}(n) &\;\longrightarrow\; \mathrm{at\_odd},\;\text{then path for } n \\
  \mathrm{Even}(n) &\;\longrightarrow\; \mathrm{at\_even},\;\text{then path for } n
\end{align*}

\paragraph{Write (insertion).}
Inserting an entry with value $v$ at key $k$ is composing a singleton
trie whose nesting mirrors the bit structure of $k$.
For example, inserting at key $5 = \mathrm{Odd}(\mathrm{Even}(\mathrm{Zero}))$:
\[
  \mathrm{trie}' = \mathrm{trie} \sqcup
  \{\mathrm{at\_odd} \mapsto \{\mathrm{at\_even} \mapsto \{\mathrm{value} \mapsto v\}\}\}
\]
Each composition creates a new immutable trie; the original is unchanged.

\paragraph{Read (static lookup).}
When the key is known statically, retrieving the entry at key $k$ is
reference-path navigation: the reference follows the same label
sequence that was used for insertion. For the same key~$5$:
\[
  \mathrm{enclosing\_scope}.\self.\mathrm{trie}.\mathrm{at\_odd}.\mathrm{at\_even}.\mathrm{value}
\]
where $\mathrm{enclosing\_scope}$ is the qualified-this scope name
containing the trie definition.
The reference mechanism traverses the nested record structure---which
\emph{is} the trie---returning the value at the final node.

\paragraph{Read (dynamic lookup).}
When the key is a BinNat value constructed at runtime via
$\mathrm{Zero}$, $\mathrm{Odd}$, and $\mathrm{Even}$, the lookup
path must be computed dynamically. This is achieved by folding the
BinNat key into a \emph{Selector}---an overlay that, given a trie,
navigates to the appropriate node and returns the stored entry.
A Selector declares the trie shape and a result slot:
\[
  \mathrm{Selector} \mapsto \{
    \mathrm{trie} \mapsto \{
      \mathrm{value} \mapsto \{\},\;
      \mathrm{at\_odd} \mapsto \{\},\;
    \mathrm{at\_even} \mapsto \{\}\},\;
  \mathrm{result} \mapsto \{\}\}
\]
Three observer callbacks build Selectors bottom-up. These callbacks
($\mathrm{on\_zero}$, $\mathrm{on\_odd}$, $\mathrm{on\_even}$) are
interpretation points in the sense of Section~5.3: they can be filled
externally by an orchestrator (CPS interpretation) or composed locally
(direct interpretation), demonstrating the CPS-agnostic property.
The $\mathrm{on\_zero}$ callback returns a Selector that reads
$.\mathrm{value}$ from the trie:
\[
  \mathrm{lookup\_on\_zero} \mapsto \mathrm{trie\_module}.\self.\mathrm{Selector} \sqcup
  \{\mathrm{result} \mapsto \mathrm{lookup\_on\_zero}.\self.\mathrm{trie}.\mathrm{value}\}
\]
The $\mathrm{on\_odd}$ callback takes a Selector
(via the $\mathrm{argument}/\mathrm{result}$ encoding of
Section~5.1) and returns a new Selector that navigates to
$.\mathrm{at\_odd}$, then delegates to the argument Selector:
\begin{align*}
  \mathrm{lookup\_on\_odd} &\mapsto \{
    \mathrm{argument} \mapsto \mathrm{trie\_module}.\self.\mathrm{Selector},\;
    \mathrm{result} \mapsto \mathrm{trie\_module}.\self.\mathrm{Selector} \sqcup \{\\
      &\qquad
      \mathrm{applied} \mapsto \mathrm{lookup\_on\_odd}.\self.\mathrm{argument} \sqcup
      \{\mathrm{trie} \mapsto \mathrm{lookup\_on\_odd}.\self.\mathrm{trie}.\mathrm{at\_odd}\},\\
      &\qquad
  \mathrm{result} \mapsto \mathrm{result}.\self.\mathrm{applied}.\mathrm{result}\}\}
\end{align*}
The $\mathrm{on\_even}$ callback is symmetric, navigating to
$.\mathrm{at\_even}$ instead.

The $\mathrm{Lookup}$ combinator folds a BinNat key with these
callbacks to obtain a Selector, then applies it to the trie:
\begin{align*}
  \mathrm{Lookup} &\mapsto \{\\
    &\quad \mathrm{key} \mapsto \mathrm{trie\_module}.\self.\mathrm{BinNat},\\
    &\quad \mathrm{trie\_input} \mapsto \{\},\\
    &\quad \mathrm{applied\_key} \mapsto \mathrm{Lookup}.\self.\mathrm{key} \sqcup \{
      \mathrm{on\_zero} \mapsto \mathrm{trie\_module}.\self.\mathrm{lookup\_on\_zero},\;
      \mathrm{on\_odd} \mapsto \mathrm{trie\_module}.\self.\mathrm{lookup\_on\_odd},\;
    \mathrm{on\_even} \mapsto \mathrm{trie\_module}.\self.\mathrm{lookup\_on\_even}\},\\
    &\quad \mathrm{applied\_selector} \mapsto
    \mathrm{Lookup}.\self.\mathrm{applied\_key}.\mathrm{result} \sqcup
    \{\mathrm{trie} \mapsto \mathrm{Lookup}.\self.\mathrm{trie\_input}\},\\
  &\quad \mathrm{result} \mapsto \mathrm{Lookup}.\self.\mathrm{applied\_selector}.\mathrm{result}\}
\end{align*}
For example, looking up key $5 = \mathrm{Odd}(\mathrm{Even}(\mathrm{Zero}))$
folds into a Selector that navigates
$.\mathrm{at\_odd}.\mathrm{at\_even}.\mathrm{value}$---the same path
as the static reference, but computed from the key value.
The entire construction uses only composition ($\sqcup$) and
references; no additional primitives are needed.

\paragraph{Delete.}
Since composition is additive---it can only merge properties, never
remove them---deletion cannot be expressed as a single composition.
Instead, deletion \emph{reconstructs} the trie, copying every subtrie
except at the targeted key.

We fold the BinNat key into a \emph{Rebuilder}---an overlay that, given
an input trie, produces a new trie as its result.
A Rebuilder declares the trie shape on both its input and output:
\[
  \mathrm{Rebuilder} \mapsto \{
    \mathrm{trie} \mapsto \{\mathrm{value}, \mathrm{at\_odd}, \mathrm{at\_even}\},\;
  \mathrm{result} \mapsto \{\mathrm{value}, \mathrm{at\_odd}, \mathrm{at\_even}\}\}
\]
Three observer callbacks build Rebuilders bottom-up.
The $\mathrm{on\_zero}$ callback returns a Rebuilder that copies the
subtries but omits the value:
\[
  \mathrm{delete\_on\_zero} \mapsto \mathrm{trie\_module}.\self.\mathrm{Rebuilder} \sqcup
  \{\mathrm{result} \mapsto \{
      \mathrm{at\_odd} \mapsto \mathrm{delete\_on\_zero}.\self.\mathrm{trie}.\mathrm{at\_odd},\;
  \mathrm{at\_even} \mapsto \mathrm{delete\_on\_zero}.\self.\mathrm{trie}.\mathrm{at\_even}\}\}
\]
The $\mathrm{result}.\mathrm{value}$ slot inherits only the empty
declaration from $\mathrm{Rebuilder}$, effectively deleting the entry.

The $\mathrm{on\_odd}$ callback takes a Rebuilder (the recursive
result for the subtrie) and returns a new Rebuilder that preserves
$\mathrm{value}$ and $\mathrm{at\_even}$, but replaces
$\mathrm{at\_odd}$ with the recursively rebuilt subtrie:
\begin{align*}
  \mathrm{delete\_on\_odd} &\mapsto \{
    \mathrm{argument} \mapsto \mathrm{trie\_module}.\self.\mathrm{Rebuilder},\;
    \mathrm{result} \mapsto \mathrm{trie\_module}.\self.\mathrm{Rebuilder} \sqcup \{\\
      &\qquad
      \mathrm{applied} \mapsto \mathrm{delete\_on\_odd}.\self.\mathrm{argument} \sqcup
      \{\mathrm{trie} \mapsto \mathrm{delete\_on\_odd}.\self.\mathrm{trie}.\mathrm{at\_odd}\},\\
      &\qquad
      \mathrm{result} \mapsto \{
        \mathrm{value} \mapsto \mathrm{delete\_on\_odd}.\self.\mathrm{trie}.\mathrm{value},\;
        \mathrm{at\_odd} \mapsto \mathrm{result}.\self.\mathrm{applied}.\mathrm{result},\;
        \mathrm{at\_even} \mapsto \mathrm{delete\_on\_odd}.\self.\mathrm{trie}.\mathrm{at\_even}
  \}\}\}
\end{align*}
The $\mathrm{on\_even}$ callback is symmetric.

The $\mathrm{Delete}$ combinator folds a BinNat key with these
callbacks to obtain a Rebuilder, then applies it to the trie:
\begin{align*}
  \mathrm{Delete} &\mapsto \{\\
    &\quad \mathrm{key} \mapsto \mathrm{trie\_module}.\self.\mathrm{BinNat},\\
    &\quad \mathrm{trie\_input} \mapsto \{\},\\
    &\quad \mathrm{applied\_key} \mapsto \mathrm{Delete}.\self.\mathrm{key} \sqcup \{
      \mathrm{on\_zero} \mapsto \mathrm{trie\_module}.\self.\mathrm{delete\_on\_zero},\;
      \mathrm{on\_odd} \mapsto \mathrm{trie\_module}.\self.\mathrm{delete\_on\_odd},\;
    \mathrm{on\_even} \mapsto \mathrm{trie\_module}.\self.\mathrm{delete\_on\_even}\},\\
    &\quad \mathrm{applied\_rebuilder} \mapsto
    \mathrm{Delete}.\self.\mathrm{applied\_key}.\mathrm{result} \sqcup
    \{\mathrm{trie} \mapsto \mathrm{Delete}.\self.\mathrm{trie\_input}\},\\
  &\quad \mathrm{result} \mapsto \mathrm{Delete}.\self.\mathrm{applied\_rebuilder}.\mathrm{result}\}
\end{align*}
Since composition is idempotent, the empty scaffolding left by the
Rebuilder schema merges harmlessly with any subsequent insertions.

\paragraph{Update.}
Updating the entry at key $k$ to a new value $v$ combines deletion
and insertion.  Because $\mathrm{Delete}$ is a combinator whose
output lives in the \emph{result} property, one must first project
out the rebuilt trie before composing with the new singleton.
In ANF style:
\begin{align*}
  \mathrm{deleted} &\mapsto \mathrm{trie\_module}.\self.\mathrm{Delete} \sqcup \{
    \mathrm{key} \mapsto k,\;
  \mathrm{trie\_input} \mapsto \mathrm{trie}\}\\
  \mathrm{result} &\mapsto \mathrm{Update}.\self.\mathrm{deleted}.\mathrm{result} \sqcup
  \{\mathrm{at\_odd} \mapsto \{\mathrm{at\_even} \mapsto
  \{\mathrm{value} \mapsto v\}\}\}
\end{align*}
where the second line uses key $5 = \mathrm{Odd}(\mathrm{Even}(\mathrm{Zero}))$
as a concrete example.
The projection $.\mathrm{result}$ is essential: $\mathrm{Delete}$
carries internal scaffolding ($\mathrm{key}$, $\mathrm{trie\_input}$,
$\mathrm{applied\_key}$, etc.), and only its $\mathrm{result}$
property holds the rebuilt trie.  This intermediate projection
prevents Update from being a single symmetric composition.

\section{Church Encoding of Standard Types}
\label{app:church}

This appendix demonstrates that standard data types can be encoded in
Overlay-Calculus without scalar types, using Church encoding. These
encodings serve as a proof of concept that a standard library can be
built entirely within Overlay-Calculus. Each encoding is presented as a
module overlay. In practical implementations, more efficient
representations (e.g., binary naturals) or FFI to host language
primitives would typically be used.
\selfcite{The implementation~\cite{mixin2025} includes}{The supplementary material includes} executable implementations of these
encodings with test cases for $\mathrm{Not}(\mathrm{True})$,
$\mathrm{And}(\mathrm{True}, \mathrm{True})$,
$\mathrm{Or}(\mathrm{False}, \mathrm{True})$, and natural number
addition, all verified against expected output snapshots.

\subsection{Boolean Module}

The boolean module contains a schema ($\mathrm{Boolean}$), constructors
($\mathrm{True}$, $\mathrm{False}$), and operations ($\mathrm{Not}$,
$\mathrm{And}$, $\mathrm{Or}$). All definitions are siblings within a
single module overlay; references use qualified this accordingly.

\begin{align*}
  \{ \quad \mathrm{boolean} \mapsto \{ \quad
      \mathrm{Boolean} &\mapsto \{\mathrm{on\_true} \mapsto \{\},\;
        \mathrm{on\_false} \mapsto \{\},\;
      \mathrm{result} \mapsto \{\}\},
      \\[6pt]
      \mathrm{True} &\mapsto \mathrm{boolean}.\self.\mathrm{Boolean} \sqcup
      \{\mathrm{result} \mapsto \mathrm{True}.\self.\mathrm{on\_true}\},
      \\
      \mathrm{False} &\mapsto \mathrm{boolean}.\self.\mathrm{Boolean} \sqcup
      \{\mathrm{result} \mapsto \mathrm{False}.\self.\mathrm{on\_false}\},
      \\[6pt]
      \mathrm{Not} &\mapsto \mathrm{boolean}.\self.\mathrm{Boolean} \sqcup \{\\
        &\qquad \mathrm{operand} \mapsto \mathrm{boolean}.\self.\mathrm{Boolean},\\
        &\qquad \mathrm{applied\_operand} \mapsto \mathrm{Not}.\self.\mathrm{operand} \sqcup \{\\
          &\qquad\qquad \mathrm{on\_true} \mapsto \mathrm{Not}.\self.\mathrm{on\_false},\\
        &\qquad\qquad \mathrm{on\_false} \mapsto \mathrm{Not}.\self.\mathrm{on\_true}\},\\
      &\qquad \mathrm{result} \mapsto \mathrm{Not}.\self.\mathrm{applied\_operand}.\mathrm{result}\},
      \\[6pt]
      \mathrm{And} &\mapsto \mathrm{boolean}.\self.\mathrm{Boolean} \sqcup \{\\
        &\qquad \mathrm{left} \mapsto \mathrm{boolean}.\self.\mathrm{Boolean},\\
        &\qquad \mathrm{right} \mapsto \mathrm{boolean}.\self.\mathrm{Boolean},\\
        &\qquad \mathrm{applied\_left} \mapsto \mathrm{And}.\self.\mathrm{left} \sqcup \{\\
          &\qquad\qquad \mathrm{on\_true} \mapsto \mathrm{And}.\self.\mathrm{right},\\
        &\qquad\qquad \mathrm{on\_false} \mapsto \mathrm{boolean}.\self.\mathrm{False}\},\\
      &\qquad \mathrm{result} \mapsto \mathrm{And}.\self.\mathrm{applied\_left}.\mathrm{result}\},
      \\[6pt]
      \mathrm{Or} &\mapsto \mathrm{boolean}.\self.\mathrm{Boolean} \sqcup \{\\
        &\qquad \mathrm{left} \mapsto \mathrm{boolean}.\self.\mathrm{Boolean},\\
        &\qquad \mathrm{right} \mapsto \mathrm{boolean}.\self.\mathrm{Boolean},\\
        &\qquad \mathrm{applied\_left} \mapsto \mathrm{Or}.\self.\mathrm{left} \sqcup \{\\
          &\qquad\qquad \mathrm{on\_true} \mapsto \mathrm{boolean}.\self.\mathrm{True},\\
        &\qquad\qquad \mathrm{on\_false} \mapsto \mathrm{Or}.\self.\mathrm{right}\},\\
      &\qquad \mathrm{result} \mapsto \mathrm{Or}.\self.\mathrm{applied\_left}.\mathrm{result}\}
  \quad \} \quad \}
\end{align*}

\paragraph{Qualified this explanation.}
Within the $\mathrm{boolean}$ module:
\begin{itemize}
  \item $\mathrm{boolean}.\self$ resolves to the module scope
    $\{\mathrm{Boolean}, \mathrm{True}, \mathrm{False},
    \mathrm{Not}, \mathrm{And}, \mathrm{Or}\}$
    after all compositions are applied.
  \item Inside True's definition:
    $\mathrm{True}.\self$ refers to True's properties after merge with Boolean
    (including $\mathrm{result}$, $\mathrm{on\_true}$, $\mathrm{on\_false}$).
    $\mathrm{boolean}.\self$ refers to the module scope.
  \item Inside Not's definition:
    $\mathrm{Not}.\self$ refers to Not's properties after merge with Boolean.
    $\mathrm{boolean}.\self$ refers to the module scope.
  \item Inside $\{\mathrm{on\_true},\; \mathrm{on\_false}\}$ nested in And/Or:
    $\mathrm{And}.\self$ (or $\mathrm{Or}.\self$) refers to the enclosing operation's scope.
    $\mathrm{boolean}.\self$ refers to the module scope,
    enabling references to sibling constructors like $\mathrm{True}$ and $\mathrm{False}$.
\end{itemize}

\paragraph{Worked example.}
Consider the expression
$\mathrm{root}.\self.\mathrm{boolean}.\mathrm{Not} \sqcup \{\mathrm{operand} \mapsto \mathrm{root}.\self.\mathrm{boolean}.\mathrm{True}\}
\sqcup \{\mathrm{on\_true} \mapsto A,\; \mathrm{on\_false} \mapsto B\}$
at the root scope level,
where $A$ and $B$ are arbitrary expressions.
The result is an object whose $\mathrm{result}$ label contains the same observable tree as $B$,
because $\mathrm{Not}$ swaps $\mathrm{on\_true}$ and $\mathrm{on\_false}$,
and $\mathrm{True}$ selects $\mathrm{on\_true}$,
which after swapping becomes $B$.

\subsection{Natural Number Module}

The natural number module contains a schema ($\mathrm{Nat}$),
constructors ($\mathrm{Zero}$, $\mathrm{Succ}$), and an operation
($\mathrm{Add}$).

\begin{align*}
  \{ \quad \mathrm{natural} \mapsto \{ \quad
      \mathrm{Nat} &\mapsto \{\mathrm{successor} \mapsto \{\mathrm{argument} \mapsto \{\},\;
        \mathrm{result} \mapsto \{\}\},\;
        \mathrm{zero} \mapsto \{\},\;
      \mathrm{result} \mapsto \{\}\},
      \\[6pt]
      \mathrm{Zero} &\mapsto \mathrm{natural}.\self.\mathrm{Nat} \sqcup
      \{\mathrm{result} \mapsto \mathrm{Zero}.\self.\mathrm{zero}\},
      \\[6pt]
      \mathrm{Succ} &\mapsto \mathrm{natural}.\self.\mathrm{Nat} \sqcup \{\\
        &\qquad \mathrm{predecessor} \mapsto \mathrm{natural}.\self.\mathrm{Nat},\\
        &\qquad \mathrm{applied\_predecessor} \mapsto \mathrm{Succ}.\self.\mathrm{predecessor} \sqcup \{\\
          &\qquad\qquad \mathrm{successor} \mapsto \mathrm{Succ}.\self.\mathrm{successor},\\
        &\qquad\qquad \mathrm{zero} \mapsto \mathrm{Succ}.\self.\mathrm{zero}\},\\
        &\qquad \mathrm{applied\_successor} \mapsto \mathrm{Succ}.\self.\mathrm{successor} \sqcup \{\\
        &\qquad\qquad \mathrm{argument} \mapsto \mathrm{Succ}.\self.\mathrm{applied\_predecessor}.\mathrm{result}\},\\
      &\qquad \mathrm{result} \mapsto \mathrm{Succ}.\self.\mathrm{applied\_successor}.\mathrm{result}\},
      \\[6pt]
      \mathrm{Add} &\mapsto \mathrm{natural}.\self.\mathrm{Nat} \sqcup \{\\
        &\qquad \mathrm{augend} \mapsto \mathrm{natural}.\self.\mathrm{Nat},\\
        &\qquad \mathrm{addend} \mapsto \mathrm{natural}.\self.\mathrm{Nat},\\
        &\qquad \mathrm{applied\_addend} \mapsto \mathrm{Add}.\self.\mathrm{addend} \sqcup \{\\
          &\qquad\qquad \mathrm{successor} \mapsto \mathrm{Add}.\self.\mathrm{successor},\\
        &\qquad\qquad \mathrm{zero} \mapsto \mathrm{Add}.\self.\mathrm{zero}\},\\
        &\qquad \mathrm{applied\_augend} \mapsto \mathrm{Add}.\self.\mathrm{augend} \sqcup \{\\
          &\qquad\qquad \mathrm{successor} \mapsto \mathrm{Add}.\self.\mathrm{successor},\\
        &\qquad\qquad \mathrm{zero} \mapsto \mathrm{Add}.\self.\mathrm{applied\_addend}.\mathrm{result}\},\\
      &\qquad \mathrm{result} \mapsto \mathrm{Add}.\self.\mathrm{applied\_augend}.\mathrm{result}\}
  \quad \} \quad \}
\end{align*}

\paragraph{Concrete example.}
At the root scope level,
$\mathrm{one} \mapsto \mathrm{root}.\self.\mathrm{natural}.\mathrm{Succ} \sqcup \{\mathrm{predecessor} \mapsto \mathrm{root}.\self.\mathrm{natural}.\mathrm{Zero}\}$.
Applying this with a Peano-style successor and zero:
$\mathrm{root}.\self.\mathrm{one} \sqcup \{\mathrm{successor} \mapsto S,\; \mathrm{zero} \mapsto Z\}$
produces an object whose $\mathrm{result}$ label contains the same observable tree as
$S \sqcup \{\mathrm{argument} \mapsto Z.\mathrm{result}\}.\mathrm{result}$,
i.e., successor applied once to zero.
Here $S$ and $Z$ are arbitrary expressions representing
a successor operation and a zero value respectively.

\end{document}
